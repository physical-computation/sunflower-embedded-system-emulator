
\section{\quad\code{ADDVALUETRACE}}\inxx{Commands,{\code{addvaluetrace}}}
\label{manpages:ADDVALUETRACE}
\label{manpages:addvaluetrace}
\vspace{-0.1in}
{\bf Description}: 	Install an address monitor to track data values.\\[1.5ex]
{\em Synopsis}:
\vspace{-0.05in}
\scriptsize
\begin{lstlisting}
   ADDVALUETRACE   <name string (string)> <base addr (hexadecimal)> <size (integer)> <onstack (Boolean)> <pcstart (hexadecimal)> <frameoffset (integer)>			
\end{lstlisting}
\normalsize
\vspace{-0.05in}


\section{\quad\code{BATTALERTFRAC}}\inxx{Commands,{\code{battalertfrac}}}
\label{manpages:BATTALERTFRAC}
\label{manpages:battalertfrac}
\vspace{-0.1in}
{\bf Description}: 	Set battery alert level fraction.\\[1.5ex]
{\em Synopsis}:
\vspace{-0.05in}
\scriptsize
\begin{lstlisting}
   BATTALERTFRAC   										
\end{lstlisting}
\normalsize
\vspace{-0.05in}


\section{\quad\code{BATTCF}}\inxx{Commands,{\code{battcf}}}
\label{manpages:BATTCF}
\label{manpages:battcf}
\vspace{-0.1in}
{\bf Description}: 	Set Battery Vrate lowpass filter capacitance.\\[1.5ex]
{\em Synopsis}:
\vspace{-0.05in}
\scriptsize
\begin{lstlisting}
   BATTCF   <capacitance in Farads (real)>																
\end{lstlisting}
\normalsize
\vspace{-0.05in}


\section{\quad\code{BATTETALUTNENTRIES}}\inxx{Commands,{\code{battetalutnentries}}}
\label{manpages:BATTETALUTNENTRIES}
\label{manpages:battetalutnentries}
\vspace{-0.1in}
{\bf Description}: 	Set number of etaLUT entries.\\[1.5ex]
{\em Synopsis}:
\vspace{-0.05in}
\scriptsize
\begin{lstlisting}
   BATTETALUTNENTRIES   <number of entries (integer)>																		
\end{lstlisting}
\normalsize
\vspace{-0.05in}


\section{\quad\code{BATTETALUT}}\inxx{Commands,{\code{battetalut}}}
\label{manpages:BATTETALUT}
\label{manpages:battetalut}
\vspace{-0.1in}
{\bf Description}: 	Set Battery etaLUT value.\\[1.5ex]
{\em Synopsis}:
\vspace{-0.05in}
\scriptsize
\begin{lstlisting}
   BATTETALUT   <LUT index (integer)> <value (real)>																		
\end{lstlisting}
\normalsize
\vspace{-0.05in}


\section{\quad\code{BATTILEAK}}\inxx{Commands,{\code{battileak}}}
\label{manpages:BATTILEAK}
\label{manpages:battileak}
\vspace{-0.1in}
{\bf Description}: 	Set Battery self-discharge current.\\[1.5ex]
{\em Synopsis}:
\vspace{-0.05in}
\scriptsize
\begin{lstlisting}
   BATTILEAK   <current in Amperes (real)>																		
\end{lstlisting}
\normalsize
\vspace{-0.05in}


\section{\quad\code{BATTINOMINAL}}\inxx{Commands,{\code{battinominal}}}
\label{manpages:BATTINOMINAL}
\label{manpages:battinominal}
\vspace{-0.1in}
{\bf Description}: 	Set Battery Inominal.\\[1.5ex]
{\em Synopsis}:
\vspace{-0.05in}
\scriptsize
\begin{lstlisting}
   BATTINOMINAL   <nominal current in Amperes (real)>																		
\end{lstlisting}
\normalsize
\vspace{-0.05in}


\section{\quad\code{BATTNODEATTACH}}\inxx{Commands,{\code{battnodeattach}}}
\label{manpages:BATTNODEATTACH}
\label{manpages:battnodeattach}
\vspace{-0.1in}
{\bf Description}: 	Attach current node to a specified battery.\\[1.5ex]
{\em Synopsis}:
\vspace{-0.05in}
\scriptsize
\begin{lstlisting}
   BATTNODEATTACH   <which battery (integer)>						
\end{lstlisting}
\normalsize
\vspace{-0.05in}


\section{\quad\code{BATTRF}}\inxx{Commands,{\code{battrf}}}
\label{manpages:BATTRF}
\label{manpages:battrf}
\vspace{-0.1in}
{\bf Description}: 	Set Battery Vrate lowpass filter resistance.\\[1.5ex]
{\em Synopsis}:
\vspace{-0.05in}
\scriptsize
\begin{lstlisting}
   BATTRF   <resistance in Ohms (real)>																
\end{lstlisting}
\normalsize
\vspace{-0.05in}


\section{\quad\code{BATTSTATS}}\inxx{Commands,{\code{battstats}}}
\label{manpages:BATTSTATS}
\label{manpages:battstats}
\vspace{-0.1in}
{\bf Description}: 	Get battery statistics.\\[1.5ex]
{\em Synopsis}:
\vspace{-0.05in}
\scriptsize
\begin{lstlisting}
   BATTSTATS   <which battery (integer)>								
\end{lstlisting}
\normalsize
\vspace{-0.05in}


\section{\quad\code{BATTVBATTLUTNENTRIES}}\inxx{Commands,{\code{battvbattlutnentries}}}
\label{manpages:BATTVBATTLUTNENTRIES}
\label{manpages:battvbattlutnentries}
\vspace{-0.1in}
{\bf Description}: 	Set number of VbattLUT entries.\\[1.5ex]
{\em Synopsis}:
\vspace{-0.05in}
\scriptsize
\begin{lstlisting}
   BATTVBATTLUTNENTRIES   <number of entries (integer)>																		
\end{lstlisting}
\normalsize
\vspace{-0.05in}


\section{\quad\code{BATTVBATTLUT}}\inxx{Commands,{\code{battvbattlut}}}
\label{manpages:BATTVBATTLUT}
\label{manpages:battvbattlut}
\vspace{-0.1in}
{\bf Description}: 	Set Battery VbattLUT value.\\[1.5ex]
{\em Synopsis}:
\vspace{-0.05in}
\scriptsize
\begin{lstlisting}
   BATTVBATTLUT   <index (integer)> <value (real)>																		
\end{lstlisting}
\normalsize
\vspace{-0.05in}


\section{\quad\code{BATTVLOSTLUTNENTRIES}}\inxx{Commands,{\code{battvlostlutnentries}}}
\label{manpages:BATTVLOSTLUTNENTRIES}
\label{manpages:battvlostlutnentries}
\vspace{-0.1in}
{\bf Description}: 	Set number of VlostLUT entries.\\[1.5ex]
{\em Synopsis}:
\vspace{-0.05in}
\scriptsize
\begin{lstlisting}
   BATTVLOSTLUTNENTRIES   <number of entries (integer)>																		
\end{lstlisting}
\normalsize
\vspace{-0.05in}


\section{\quad\code{BATTVLOSTLUT}}\inxx{Commands,{\code{battvlostlut}}}
\label{manpages:BATTVLOSTLUT}
\label{manpages:battvlostlut}
\vspace{-0.1in}
{\bf Description}: 	Set Battery VlostLUT value.\\[1.5ex]
{\em Synopsis}:
\vspace{-0.05in}
\scriptsize
\begin{lstlisting}
   BATTVLOSTLUT   <index (integer)> <value (real)>																		
\end{lstlisting}
\normalsize
\vspace{-0.05in}


\section{\quad\code{BPTDEL}}\inxx{Commands,{\code{bptdel}}}
\label{manpages:BPTDEL}
\label{manpages:bptdel}
\vspace{-0.1in}
{\bf Description}: 	Delete breakpoint.\\[1.5ex]
{\em Synopsis}:
\vspace{-0.05in}
\scriptsize
\begin{lstlisting}
   BPTDEL   <breakpoint ID (integer)>																				
\end{lstlisting}
\normalsize
\vspace{-0.05in}


\section{\quad\code{BPTLS}}\inxx{Commands,{\code{bptls}}}
\label{manpages:BPTLS}
\label{manpages:bptls}
\vspace{-0.1in}
{\bf Description}: 	List breakpoints and their IDs.\\[1.5ex]
{\em Synopsis}:
\vspace{-0.05in}
\scriptsize
\begin{lstlisting}
   BPTLS   																					
\end{lstlisting}
\normalsize
\vspace{-0.05in}


\section{\quad\code{BPT}}\inxx{Commands,{\code{bpt}}}
\label{manpages:BPT}
\label{manpages:bpt}
\vspace{-0.1in}
{\bf Description}: 	Set breakpoint.\\[1.5ex]
{\em Synopsis}:
\vspace{-0.05in}
\scriptsize
\begin{lstlisting}
   BPT    'cycles' <ncycles on current node (integer)> | 'instrs' <ninstrs on current node (integer)> | 'sensorreading' <which sensor (integer)> <value (real)> | 'globaltime' <global time in picoseconds (integer)>	
\end{lstlisting}
\normalsize
\vspace{-0.05in}


\section{\quad\code{CACHEINIT}}\inxx{Commands,{\code{cacheinit}}}
\label{manpages:CACHEINIT}
\label{manpages:cacheinit}
\vspace{-0.1in}
{\bf Description}: 	Initialise cache.\\[1.5ex]
{\em Synopsis}:
\vspace{-0.05in}
\scriptsize
\begin{lstlisting}
   CACHEINIT   <cache size (integer)> <block size (integer)> <set associativity (integer)>			
\end{lstlisting}
\normalsize
\vspace{-0.05in}


\section{\quad\code{CACHEOFF}}\inxx{Commands,{\code{cacheoff}}}
\label{manpages:CACHEOFF}
\label{manpages:cacheoff}
\vspace{-0.1in}
{\bf Description}: 	Deactivate cache.\\[1.5ex]
{\em Synopsis}:
\vspace{-0.05in}
\scriptsize
\begin{lstlisting}
   CACHEOFF    												
\end{lstlisting}
\normalsize
\vspace{-0.05in}


\section{\quad\code{CACHESTATS}}\inxx{Commands,{\code{cachestats}}}
\label{manpages:CACHESTATS}
\label{manpages:cachestats}
\vspace{-0.1in}
{\bf Description}: 	Retrieve cache access statistics.\\[1.5ex]
{\em Synopsis}:
\vspace{-0.05in}
\scriptsize
\begin{lstlisting}
   CACHESTATS   										
\end{lstlisting}
\normalsize
\vspace{-0.05in}


\section{\quad\code{CA}}\inxx{Commands,{\code{ca}}}
\label{manpages:CA}
\label{manpages:ca}
\vspace{-0.1in}
{\bf Description}: 	Set simulator in cycle-accurate mode.\\[1.5ex]
{\em Synopsis}:
\vspace{-0.05in}
\scriptsize
\begin{lstlisting}
   CA   									
\end{lstlisting}
\normalsize
\vspace{-0.05in}


\section{\quad\code{CD}}\inxx{Commands,{\code{cd}}}
\label{manpages:CD}
\label{manpages:cd}
\vspace{-0.1in}
{\bf Description}: 	Change current working directory.\\[1.5ex]
{\em Synopsis}:
\vspace{-0.05in}
\scriptsize
\begin{lstlisting}
   CD   <path (string)>																			
\end{lstlisting}
\normalsize
\vspace{-0.05in}


\section{\quad\code{CLOCKINTR}}\inxx{Commands,{\code{clockintr}}}
\label{manpages:CLOCKINTR}
\label{manpages:clockintr}
\vspace{-0.1in}
{\bf Description}: 	Toggle enabling clock interrupts.\\[1.5ex]
{\em Synopsis}:
\vspace{-0.05in}
\scriptsize
\begin{lstlisting}
   CLOCKINTR   <flag (Boolean)>								
\end{lstlisting}
\normalsize
\vspace{-0.05in}


\section{\quad\code{CONT}}\inxx{Commands,{\code{cont}}}
\label{manpages:CONT}
\label{manpages:cont}
\vspace{-0.1in}
{\bf Description}: 	Continue execution while PC is not equal to specified PC.\\[1.5ex]
{\em Synopsis}:
\vspace{-0.05in}
\scriptsize
\begin{lstlisting}
   CONT   <until PC (hexadecimal)>				
\end{lstlisting}
\normalsize
\vspace{-0.05in}


\section{\quad\code{C}}\inxx{Commands,{\code{c}}}
\label{manpages:C}
\label{manpages:c}
\vspace{-0.1in}
{\bf Description}: 	Synonym for CACHESTATS.\\[1.5ex]
{\em Synopsis}:
\vspace{-0.05in}
\scriptsize
\begin{lstlisting}
   C   											
\end{lstlisting}
\normalsize
\vspace{-0.05in}


\section{\quad\code{DEFNDIST}}\inxx{Commands,{\code{defndist}}}
\label{manpages:DEFNDIST}
\label{manpages:defndist}
\vspace{-0.1in}
{\bf Description}: 	Define a discrete probability measure as a set of basis value probability tuples.\\[1.5ex]
{\em Synopsis}:
\vspace{-0.05in}
\scriptsize
\begin{lstlisting}
   DEFNDIST   <list of basis value> (integer list) <list of probabilities>	(real list)						
\end{lstlisting}
\normalsize
\vspace{-0.05in}


\section{\quad\code{DELVALUETRACE}}\inxx{Commands,{\code{delvaluetrace}}}
\label{manpages:DELVALUETRACE}
\label{manpages:delvaluetrace}
\vspace{-0.1in}
{\bf Description}: 	Delete an installed address monitor for tracking data values.\\[1.5ex]
{\em Synopsis}:
\vspace{-0.05in}
\scriptsize
\begin{lstlisting}
   DELVALUETRACE   <name string (string)> <base addr (hexadecimal)> <size (integer)> <onstack (Boolean)> <pcstart (hexadecimal)> <frameoffset (integer)>	
\end{lstlisting}
\normalsize
\vspace{-0.05in}


\section{\quad\code{D}}\inxx{Commands,{\code{d}}}
\label{manpages:D}
\label{manpages:d}
\vspace{-0.1in}
{\bf Description}: 	Synonym for DUMPALL.\\[1.5ex]
{\em Synopsis}:
\vspace{-0.05in}
\scriptsize
\begin{lstlisting}
   D   <filename (string)> <tag (string)> <prefix (string)>																
\end{lstlisting}
\normalsize
\vspace{-0.05in}


\section{\quad\code{DUMPALL}}\inxx{Commands,{\code{dumpall}}}
\label{manpages:DUMPALL}
\label{manpages:dumpall}
\vspace{-0.1in}
{\bf Description}: 	Dump the State structure info for all nodes to the file using given tag and prefix.\\[1.5ex]
{\em Synopsis}:
\vspace{-0.05in}
\scriptsize
\begin{lstlisting}
   DUMPALL   <filename (string)> <tag (string)> <prefix (string)>								
\end{lstlisting}
\normalsize
\vspace{-0.05in}


\section{\quad\code{DUMPDISTR}}\inxx{Commands,{\code{dumpdistr}}}
\label{manpages:DUMPDISTR}
\label{manpages:dumpdistr}
\vspace{-0.1in}
{\bf Description}: 	Show the number of times each instruction was run.\\[1.5ex]
{\em Synopsis}:
\vspace{-0.05in}
\scriptsize
\begin{lstlisting}
   DUMPDISTR   								
\end{lstlisting}
\normalsize
\vspace{-0.05in}


\section{\quad\code{DUMPMEM}}\inxx{Commands,{\code{dumpmem}}}
\label{manpages:DUMPMEM}
\label{manpages:dumpmem}
\vspace{-0.1in}
{\bf Description}: 	Show contents of memory.\\[1.5ex]
{\em Synopsis}:
\vspace{-0.05in}
\scriptsize
\begin{lstlisting}
   DUMPMEM   <start mem address (hexadecimal)> <end mem address (hexadecimal)>			
\end{lstlisting}
\normalsize
\vspace{-0.05in}


\section{\quad\code{DUMPPIPE}}\inxx{Commands,{\code{dumppipe}}}
\label{manpages:DUMPPIPE}
\label{manpages:dumppipe}
\vspace{-0.1in}
{\bf Description}: 	Show the contents of the pipeline stages.\\[1.5ex]
{\em Synopsis}:
\vspace{-0.05in}
\scriptsize
\begin{lstlisting}
   DUMPPIPE   									
\end{lstlisting}
\normalsize
\vspace{-0.05in}


\section{\quad\code{DUMPREGS}}\inxx{Commands,{\code{dumpregs}}}
\label{manpages:DUMPREGS}
\label{manpages:dumpregs}
\vspace{-0.1in}
{\bf Description}: 	Show the contents of the general purpose registers.\\[1.5ex]
{\em Synopsis}:
\vspace{-0.05in}
\scriptsize
\begin{lstlisting}
   DUMPREGS   							
\end{lstlisting}
\normalsize
\vspace{-0.05in}


\section{\quad\code{DUMPSYSREGS}}\inxx{Commands,{\code{dumpsysregs}}}
\label{manpages:DUMPSYSREGS}
\label{manpages:dumpsysregs}
\vspace{-0.1in}
{\bf Description}: 	Show the contents of the system registers.\\[1.5ex]
{\em Synopsis}:
\vspace{-0.05in}
\scriptsize
\begin{lstlisting}
   DUMPSYSREGS   									
\end{lstlisting}
\normalsize
\vspace{-0.05in}


\section{\quad\code{DUMPTLB}}\inxx{Commands,{\code{dumptlb}}}
\label{manpages:DUMPTLB}
\label{manpages:dumptlb}
\vspace{-0.1in}
{\bf Description}: 	Display all TLB entries.\\[1.5ex]
{\em Synopsis}:
\vspace{-0.05in}
\scriptsize
\begin{lstlisting}
   DUMPTLB   																						
\end{lstlisting}
\normalsize
\vspace{-0.05in}


\section{\quad\code{DYNINSTR}}\inxx{Commands,{\code{dyninstr}}}
\label{manpages:DYNINSTR}
\label{manpages:dyninstr}
\vspace{-0.1in}
{\bf Description}: 	Display number of instructions executed.\\[1.5ex]
{\em Synopsis}:
\vspace{-0.05in}
\scriptsize
\begin{lstlisting}
   DYNINSTR   									
\end{lstlisting}
\normalsize
\vspace{-0.05in}


\section{\quad\code{EBATTINTR}}\inxx{Commands,{\code{ebattintr}}}
\label{manpages:EBATTINTR}
\label{manpages:ebattintr}
\vspace{-0.1in}
{\bf Description}: 	Toggle enable low battery level interrupts.\\[1.5ex]
{\em Synopsis}:
\vspace{-0.05in}
\scriptsize
\begin{lstlisting}
   EBATTINTR   <flag (Boolean)>							
\end{lstlisting}
\normalsize
\vspace{-0.05in}


\section{\quad\code{EFAULTS}}\inxx{Commands,{\code{efaults}}}
\label{manpages:EFAULTS}
\label{manpages:efaults}
\vspace{-0.1in}
{\bf Description}: 	Enable interuppt when too many faults occur.\\[1.5ex]
{\em Synopsis}:
\vspace{-0.05in}
\scriptsize
\begin{lstlisting}
   EFAULTS   								
\end{lstlisting}
\normalsize
\vspace{-0.05in}


\section{\quad\code{FF}}\inxx{Commands,{\code{ff}}}
\label{manpages:FF}
\label{manpages:ff}
\vspace{-0.1in}
{\bf Description}: 	Set simulator in fast functional mode.\\[1.5ex]
{\em Synopsis}:
\vspace{-0.05in}
\scriptsize
\begin{lstlisting}
   FF   									
\end{lstlisting}
\normalsize
\vspace{-0.05in}


\section{\quad\code{FILE2NETSEG}}\inxx{Commands,{\code{file2netseg}}}
\label{manpages:FILE2NETSEG}
\label{manpages:file2netseg}
\vspace{-0.1in}
{\bf Description}: 	Connect file to netseg.\\[1.5ex]
{\em Synopsis}:
\vspace{-0.05in}
\scriptsize
\begin{lstlisting}
   FILE2NETSEG   <file (string)>	<netseg (integer)>																		
\end{lstlisting}
\normalsize
\vspace{-0.05in}


\section{\quad\code{FLTTHRESH}}\inxx{Commands,{\code{fltthresh}}}
\label{manpages:FLTTHRESH}
\label{manpages:fltthresh}
\vspace{-0.1in}
{\bf Description}: 	Set threshold for EFAULTS.\\[1.5ex]
{\em Synopsis}:
\vspace{-0.05in}
\scriptsize
\begin{lstlisting}
   FLTTHRESH   <threshold (integer)>								
\end{lstlisting}
\normalsize
\vspace{-0.05in}


\section{\quad\code{FORCEAVGPWR}}\inxx{Commands,{\code{forceavgpwr}}}
\label{manpages:FORCEAVGPWR}
\label{manpages:forceavgpwr}
\vspace{-0.1in}
{\bf Description}: 	Bypass ILPA analysis and set avg pwr consumption.\\[1.5ex]
{\em Synopsis}:
\vspace{-0.05in}
\scriptsize
\begin{lstlisting}
   FORCEAVGPWR   <avg pwr in Watts (real)> <sleep pwr in Watts (real)>													
\end{lstlisting}
\normalsize
\vspace{-0.05in}


\section{\quad\code{GETRANDOMSEED}}\inxx{Commands,{\code{getrandomseed}}}
\label{manpages:GETRANDOMSEED}
\label{manpages:getrandomseed}
\vspace{-0.1in}
{\bf Description}: 	Query seed used to initialize random number generation system useful for reinitializing generator to same seed for reproducibility.\\[1.5ex]
{\em Synopsis}:
\vspace{-0.05in}
\scriptsize
\begin{lstlisting}
   GETRANDOMSEED   								
\end{lstlisting}
\normalsize
\vspace{-0.05in}


\section{\quad\code{HELP}}\inxx{Commands,{\code{help}}}
\label{manpages:HELP}
\label{manpages:help}
\vspace{-0.1in}
{\bf Description}: 	Print list of commands.\\[1.5ex]
{\em Synopsis}:
\vspace{-0.05in}
\scriptsize
\begin{lstlisting}
   HELP   											
\end{lstlisting}
\normalsize
\vspace{-0.05in}


\section{\quad\code{HWSEEREG}}\inxx{Commands,{\code{hwseereg}}}
\label{manpages:HWSEEREG}
\label{manpages:hwseereg}
\vspace{-0.1in}
{\bf Description}: 	Register a hardware structure or part thereof for inducement of SEEs.\\[1.5ex]
{\em Synopsis}:
\vspace{-0.05in}
\scriptsize
\begin{lstlisting}
   HWSEEREG    <structure name (string)> <actual bits (integer)> <logical bits (integer)> <bit offset (integer)>				
\end{lstlisting}
\normalsize
\vspace{-0.05in}


\section{\quad\code{IGN}}\inxx{Commands,{\code{ign}}}
\label{manpages:IGN}
\label{manpages:ign}
\vspace{-0.1in}
{\bf Description}: 	Ignore node fatalities and continue sim without pausing.\\[1.5ex]
{\em Synopsis}:
\vspace{-0.05in}
\scriptsize
\begin{lstlisting}
   IGN   <flag (Boolean)>																
\end{lstlisting}
\normalsize
\vspace{-0.05in}


\section{\quad\code{INITRANDTABLE}}\inxx{Commands,{\code{initrandtable}}}
\label{manpages:INITRANDTABLE}
\label{manpages:initrandtable}
\vspace{-0.1in}
{\bf Description}: 	Set or change node location.\\[1.5ex]
{\em Synopsis}:
\vspace{-0.05in}
\scriptsize
\begin{lstlisting}
   INITRANDTABLE   <distname (string)> <pfun name (string)> <basis min (real)> <basis max (real)> <granularity (real)> <p1 (real)> <p2 (real)> <p3 (real)> <p4 (real)>			
\end{lstlisting}
\normalsize
\vspace{-0.05in}


\section{\quad\code{INITSEESTATE}}\inxx{Commands,{\code{initseestate}}}
\label{manpages:INITSEESTATE}
\label{manpages:initseestate}
\vspace{-0.1in}
{\bf Description}: 	Initialize SEE function and parameter state.\\[1.5ex]
{\em Synopsis}:
\vspace{-0.05in}
\scriptsize
\begin{lstlisting}
   INITSEESTATE    <loc pfun (string)> <loc p1 (real)> <loc p2 (real)> <loc p3 (real)> <loc p4 (real)> <bit pfun (string)> <bit p1 (real)> <bit p2 (real)> <bit p3 (real)> <bit p4 (real)> <duration pfun (string)> <dur p1 (real)> <dur p2 (real)> <dur p3 (real)> <dur p4 (real)>		
\end{lstlisting}
\normalsize
\vspace{-0.05in}


\section{\quad\code{LISTRVARS}}\inxx{Commands,{\code{listrvars}}}
\label{manpages:LISTRVARS}
\label{manpages:listrvars}
\vspace{-0.1in}
{\bf Description}: 	List all structures that can be treated as rvars.\\[1.5ex]
{\em Synopsis}:
\vspace{-0.05in}
\scriptsize
\begin{lstlisting}
   LISTRVARS   								
\end{lstlisting}
\normalsize
\vspace{-0.05in}


\section{\quad\code{LOADMAPFILE}}\inxx{Commands,{\code{loadmapfile}}}
\label{manpages:LOADMAPFILE}
\label{manpages:loadmapfile}
\vspace{-0.1in}
{\bf Description}: 	Load map file.\\[1.5ex]
{\em Synopsis}:
\vspace{-0.05in}
\scriptsize
\begin{lstlisting}
   LOADMAPFILE   								
\end{lstlisting}
\normalsize
\vspace{-0.05in}


\section{\quad\code{LOAD}}\inxx{Commands,{\code{load}}}
\label{manpages:LOAD}
\label{manpages:load}
\vspace{-0.1in}
{\bf Description}: 	Load a script file.\\[1.5ex]
{\em Synopsis}:
\vspace{-0.05in}
\scriptsize
\begin{lstlisting}
   LOAD   <filename (string)>										
\end{lstlisting}
\normalsize
\vspace{-0.05in}


\section{\quad\code{LOCSTATS}}\inxx{Commands,{\code{locstats}}}
\label{manpages:LOCSTATS}
\label{manpages:locstats}
\vspace{-0.1in}
{\bf Description}: 	Show node's current location in three-dimentional space.\\[1.5ex]
{\em Synopsis}:
\vspace{-0.05in}
\scriptsize
\begin{lstlisting}
   LOCSTATS   							
\end{lstlisting}
\normalsize
\vspace{-0.05in}


\section{\quad\code{L}}\inxx{Commands,{\code{l}}}
\label{manpages:L}
\label{manpages:l}
\vspace{-0.1in}
{\bf Description}: 	Synonym for LOCSTATS.\\[1.5ex]
{\em Synopsis}:
\vspace{-0.05in}
\scriptsize
\begin{lstlisting}
   L   											
\end{lstlisting}
\normalsize
\vspace{-0.05in}


\section{\quad\code{MALLOCDEBUG}}\inxx{Commands,{\code{mallocdebug}}}
\label{manpages:MALLOCDEBUG}
\label{manpages:mallocdebug}
\vspace{-0.1in}
{\bf Description}: 	Display malloc stats.\\[1.5ex]
{\em Synopsis}:
\vspace{-0.05in}
\scriptsize
\begin{lstlisting}
   MALLOCDEBUG   																						
\end{lstlisting}
\normalsize
\vspace{-0.05in}


\section{\quad\code{MAN}}\inxx{Commands,{\code{man}}}
\label{manpages:MAN}
\label{manpages:man}
\vspace{-0.1in}
{\bf Description}: 	Print synopsis for command usage.\\[1.5ex]
{\em Synopsis}:
\vspace{-0.05in}
\scriptsize
\begin{lstlisting}
   MAN   <command name>								
\end{lstlisting}
\normalsize
\vspace{-0.05in}


\section{\quad\code{MMAP}}\inxx{Commands,{\code{mmap}}}
\label{manpages:MMAP}
\label{manpages:mmap}
\vspace{-0.1in}
{\bf Description}: 	Map memory of one simulated node into another.\\[1.5ex]
{\em Synopsis}:
\vspace{-0.05in}
\scriptsize
\begin{lstlisting}
   MMAP   <source (integer)> <destination (integer)>			
\end{lstlisting}
\normalsize
\vspace{-0.05in}


\section{\quad\code{NANOPAUSE}}\inxx{Commands,{\code{nanopause}}}
\label{manpages:NANOPAUSE}
\label{manpages:nanopause}
\vspace{-0.1in}
{\bf Description}: 	Pause the simulation for arg nanoseconds.\\[1.5ex]
{\em Synopsis}:
\vspace{-0.05in}
\scriptsize
\begin{lstlisting}
   NANOPAUSE   <duration of pause in nanoseconds (integer)>				
\end{lstlisting}
\normalsize
\vspace{-0.05in}


\section{\quad\code{ND}}\inxx{Commands,{\code{nd}}}
\label{manpages:ND}
\label{manpages:nd}
\vspace{-0.1in}
{\bf Description}: 	Synonym for NETDEBUG.\\[1.5ex]
{\em Synopsis}:
\vspace{-0.05in}
\scriptsize
\begin{lstlisting}
   ND   											
\end{lstlisting}
\normalsize
\vspace{-0.05in}


\section{\quad\code{NETCORREL}}\inxx{Commands,{\code{netcorrel}}}
\label{manpages:NETCORREL}
\label{manpages:netcorrel}
\vspace{-0.1in}
{\bf Description}: 	Specify correlation coefficient between failure of a network segment and failure of an IFC on a node @@NOTE that it is not using the current node so we can specify in a matrix-like form@@.\\[1.5ex]
{\em Synopsis}:
\vspace{-0.05in}
\scriptsize
\begin{lstlisting}
   NETCORREL   <which seg (integer)> <which node (integer)> <coefficient (real)>	
\end{lstlisting}
\normalsize
\vspace{-0.05in}


\section{\quad\code{NETDEBUG}}\inxx{Commands,{\code{netdebug}}}
\label{manpages:NETDEBUG}
\label{manpages:netdebug}
\vspace{-0.1in}
{\bf Description}: 	Show debugging information about the simulated network interface.\\[1.5ex]
{\em Synopsis}:
\vspace{-0.05in}
\scriptsize
\begin{lstlisting}
   NETDEBUG   						
\end{lstlisting}
\normalsize
\vspace{-0.05in}


\section{\quad\code{NETNEWSEG}}\inxx{Commands,{\code{netnewseg}}}
\label{manpages:NETNEWSEG}
\label{manpages:netnewseg}
\vspace{-0.1in}
{\bf Description}: 	Add a new network segment to simulation.\\[1.5ex]
{\em Synopsis}:
\vspace{-0.05in}
\scriptsize
\begin{lstlisting}
   NETNEWSEG   <which (if exists) (integer)> <frame bits (integer)> <propagation speed (integer)> <bitrate (integer)> <medium width (integer)> <link failure probability distribution (string)> <link failure distribution mu (real)> <link failure probability distribution sigma (real)> <link failure probability distribution lambda (real)> <link failure duration distribution (string)> <link failure duration distribution mu (real)> <link failure duration distribution sigma (real)> <link failure duration distribution lambda (real)>	
\end{lstlisting}
\normalsize
\vspace{-0.05in}


\section{\quad\code{NETNODENEWIFC}}\inxx{Commands,{\code{netnodenewifc}}}
\label{manpages:NETNODENEWIFC}
\label{manpages:netnodenewifc}
\vspace{-0.1in}
{\bf Description}: 	Add a new IFC to current node frame bits and segno are set at attach time.\\[1.5ex]
{\em Synopsis}:
\vspace{-0.05in}
\scriptsize
\begin{lstlisting}
   NETNODENEWIFC   <ifc num (if valid) (integer)> <tx pwr (watts) (real)> <rx pwr (watts) (real)> <idle pwr (watts) (real)> <listen pwr (watts) (real)> <fail distribution (string)> <fail distribution mu (real)> <fail distribution sigma (real)> <fail distribution lambda (real)> <transmit FIFO size (integer)> <receive FIFO size (integer)> 
\end{lstlisting}
\normalsize
\vspace{-0.05in}


\section{\quad\code{NETSEG2FILE}}\inxx{Commands,{\code{netseg2file}}}
\label{manpages:NETSEG2FILE}
\label{manpages:netseg2file}
\vspace{-0.1in}
{\bf Description}: 	Connect netseg to file.\\[1.5ex]
{\em Synopsis}:
\vspace{-0.05in}
\scriptsize
\begin{lstlisting}
   NETSEG2FILE   <netseg (integer)> <file (string)>																		
\end{lstlisting}
\normalsize
\vspace{-0.05in}


\section{\quad\code{NETSEGDELETE}}\inxx{Commands,{\code{netsegdelete}}}
\label{manpages:NETSEGDELETE}
\label{manpages:netsegdelete}
\vspace{-0.1in}
{\bf Description}: 	Disable a specified network segment.\\[1.5ex]
{\em Synopsis}:
\vspace{-0.05in}
\scriptsize
\begin{lstlisting}
   NETSEGDELETE   <which segment (integer)>																		
\end{lstlisting}
\normalsize
\vspace{-0.05in}


\section{\quad\code{NETSEGFAILDURMAX}}\inxx{Commands,{\code{netsegfaildurmax}}}
\label{manpages:NETSEGFAILDURMAX}
\label{manpages:netsegfaildurmax}
\vspace{-0.1in}
{\bf Description}: 	Set maximum network segment failure duration in clock cycles though actual failure duration is determined by probability distribution.\\[1.5ex]
{\em Synopsis}:
\vspace{-0.05in}
\scriptsize
\begin{lstlisting}
   NETSEGFAILDURMAX   <duration (integer)>						
\end{lstlisting}
\normalsize
\vspace{-0.05in}


\section{\quad\code{NETSEGFAILPROBFN}}\inxx{Commands,{\code{netsegfailprobfn}}}
\label{manpages:NETSEGFAILPROBFN}
\label{manpages:netsegfailprobfn}
\vspace{-0.1in}
{\bf Description}: 	Specify Netseg failure Probability Distribution Function (fxn of time).\\[1.5ex]
{\em Synopsis}:
\vspace{-0.05in}
\scriptsize
\begin{lstlisting}
   NETSEGFAILPROBFN   <expression in terms of constants and 'pow(a
\end{lstlisting}
\normalsize
\vspace{-0.05in}


\section{\quad\code{NETSEGFAILPROB}}\inxx{Commands,{\code{netsegfailprob}}}
\label{manpages:NETSEGFAILPROB}
\label{manpages:netsegfailprob}
\vspace{-0.1in}
{\bf Description}: 	Set probability of failure for a setseg.\\[1.5ex]
{\em Synopsis}:
\vspace{-0.05in}
\scriptsize
\begin{lstlisting}
   NETSEGFAILPROB   <which segment (integer)> <probability (real)>															
\end{lstlisting}
\normalsize
\vspace{-0.05in}


\section{\quad\code{NETSEGNICATTACH}}\inxx{Commands,{\code{netsegnicattach}}}
\label{manpages:NETSEGNICATTACH}
\label{manpages:netsegnicattach}
\vspace{-0.1in}
{\bf Description}: 	Attach a current node's IFC to a network segment.\\[1.5ex]
{\em Synopsis}:
\vspace{-0.05in}
\scriptsize
\begin{lstlisting}
   NETSEGNICATTACH   <which IFC (integer)>	<which segment (integer)>													
\end{lstlisting}
\normalsize
\vspace{-0.05in}


\section{\quad\code{NETSEGPROPMODEL}}\inxx{Commands,{\code{netsegpropmodel}}}
\label{manpages:NETSEGPROPMODEL}
\label{manpages:netsegpropmodel}
\vspace{-0.1in}
{\bf Description}: 	Associate a network segment with a signal propagation model.\\[1.5ex]
{\em Synopsis}:
\vspace{-0.05in}
\scriptsize
\begin{lstlisting}
   NETSEGPROPMODEL   <netseg ID (integer)> <sigsrc ID (integer)> <minimum SNR (real)>										
\end{lstlisting}
\normalsize
\vspace{-0.05in}


\section{\quad\code{NEWBATT}}\inxx{Commands,{\code{newbatt}}}
\label{manpages:NEWBATT}
\label{manpages:newbatt}
\vspace{-0.1in}
{\bf Description}: 	New battery\\[1.5ex]
{\em Synopsis}:
\vspace{-0.05in}
\scriptsize
\begin{lstlisting}
   NEWBATT   <ID (integer)> <capacity in mAh (integer)>								
\end{lstlisting}
\normalsize
\vspace{-0.05in}


\section{\quad\code{NEWNODE}}\inxx{Commands,{\code{newnode}}}
\label{manpages:NEWNODE}
\label{manpages:newnode}
\vspace{-0.1in}
{\bf Description}: 	Create a new node (simulated system).\\[1.5ex]
{\em Synopsis}:
\vspace{-0.05in}
\scriptsize
\begin{lstlisting}
   NEWNODE   <type=superH|riscv|msp430 (string)> [<x location (real)> <y location (real)> <z location (real)>] [<trajectory file name (string)> <loopsamples (Boolean)> <picoseconds per trajectory sample (integer)>]	
\end{lstlisting}
\normalsize
\vspace{-0.05in}


\section{\quad\code{NI}}\inxx{Commands,{\code{ni}}}
\label{manpages:NI}
\label{manpages:ni}
\vspace{-0.1in}
{\bf Description}: 	Synonym for DYNINSTR.\\[1.5ex]
{\em Synopsis}:
\vspace{-0.05in}
\scriptsize
\begin{lstlisting}
   NI   											
\end{lstlisting}
\normalsize
\vspace{-0.05in}


\section{\quad\code{NODEFAILDURMAX}}\inxx{Commands,{\code{nodefaildurmax}}}
\label{manpages:NODEFAILDURMAX}
\label{manpages:nodefaildurmax}
\vspace{-0.1in}
{\bf Description}: 	Set maximum node failure duration in clock cycles though actual failure duration is determined by probability distribution.\\[1.5ex]
{\em Synopsis}:
\vspace{-0.05in}
\scriptsize
\begin{lstlisting}
   NODEFAILDURMAX   <duration (integer)>							
\end{lstlisting}
\normalsize
\vspace{-0.05in}


\section{\quad\code{NODEFAILPROBFN}}\inxx{Commands,{\code{nodefailprobfn}}}
\label{manpages:NODEFAILPROBFN}
\label{manpages:nodefailprobfn}
\vspace{-0.1in}
{\bf Description}: 	Specify Node failure Probability Distribution Function (fxn of time).\\[1.5ex]
{\em Synopsis}:
\vspace{-0.05in}
\scriptsize
\begin{lstlisting}
   NODEFAILPROBFN   <expression in terms of constants and 'pow(a
\end{lstlisting}
\normalsize
\vspace{-0.05in}


\section{\quad\code{NODEFAILPROB}}\inxx{Commands,{\code{nodefailprob}}}
\label{manpages:NODEFAILPROB}
\label{manpages:nodefailprob}
\vspace{-0.1in}
{\bf Description}: 	Set probability of failure for current node.\\[1.5ex]
{\em Synopsis}:
\vspace{-0.05in}
\scriptsize
\begin{lstlisting}
   NODEFAILPROB   <probability (real)>																	
\end{lstlisting}
\normalsize
\vspace{-0.05in}


\section{\quad\code{NODETACH}}\inxx{Commands,{\code{nodetach}}}
\label{manpages:NODETACH}
\label{manpages:nodetach}
\vspace{-0.1in}
{\bf Description}: 	Set whether new thread should be spawned on a ON command.\\[1.5ex]
{\em Synopsis}:
\vspace{-0.05in}
\scriptsize
\begin{lstlisting}
   NODETACH   <flag (Boolean)>					
\end{lstlisting}
\normalsize
\vspace{-0.05in}


\section{\quad\code{N}}\inxx{Commands,{\code{n}}}
\label{manpages:N}
\label{manpages:n}
\vspace{-0.1in}
{\bf Description}: 	Step through simulation for a number (default 1) of cycles.\\[1.5ex]
{\em Synopsis}:
\vspace{-0.05in}
\scriptsize
\begin{lstlisting}
   N   [# cycles] (integer)				
\end{lstlisting}
\normalsize
\vspace{-0.05in}


\section{\quad\code{NUMAREGION}}\inxx{Commands,{\code{numaregion}}}
\label{manpages:NUMAREGION}
\label{manpages:numaregion}
\vspace{-0.1in}
{\bf Description}: 	Specify a memory access latency and a node mapping (can only map into destination RAM) for an address range for a private mapping.\\[1.5ex]
{\em Synopsis}:
\vspace{-0.05in}
\scriptsize
\begin{lstlisting}
   NUMAREGION   <name string (string)> <start address (inclusive) (hexadecimal)> <end address (non-inclusive) (hexadecimal)> <local read latency in cycles (integer)> <local write latency in cycles (integer)> <remote read latency in cycles (integer)> <remote write latency in cycles (integer)> <Map ID (integer)> <Map offset (integer)> <private flag (Boolean)> 
\end{lstlisting}
\normalsize
\vspace{-0.05in}


\section{\quad\code{NUMASETMAPID}}\inxx{Commands,{\code{numasetmapid}}}
\label{manpages:NUMASETMAPID}
\label{manpages:numasetmapid}
\vspace{-0.1in}
{\bf Description}: 	Change the mapid for nth map table entry on all nodes to i.\\[1.5ex]
{\em Synopsis}:
\vspace{-0.05in}
\scriptsize
\begin{lstlisting}
   NUMASETMAPID   <n (integer)> <i (integer)>															
\end{lstlisting}
\normalsize
\vspace{-0.05in}


\section{\quad\code{NUMASTATSALL}}\inxx{Commands,{\code{numastatsall}}}
\label{manpages:NUMASTATSALL}
\label{manpages:numastatsall}
\vspace{-0.1in}
{\bf Description}: 	Display access statistics for all NUMA regions for all nodes.\\[1.5ex]
{\em Synopsis}:
\vspace{-0.05in}
\scriptsize
\begin{lstlisting}
   NUMASTATSALL   																	
\end{lstlisting}
\normalsize
\vspace{-0.05in}


\section{\quad\code{NUMASTATS}}\inxx{Commands,{\code{numastats}}}
\label{manpages:NUMASTATS}
\label{manpages:numastats}
\vspace{-0.1in}
{\bf Description}: 	Display access statistics for all NUMA regions for current node.\\[1.5ex]
{\em Synopsis}:
\vspace{-0.05in}
\scriptsize
\begin{lstlisting}
   NUMASTATS   																	
\end{lstlisting}
\normalsize
\vspace{-0.05in}


\section{\quad\code{OFF}}\inxx{Commands,{\code{off}}}
\label{manpages:OFF}
\label{manpages:off}
\vspace{-0.1in}
{\bf Description}: 	Turn the simulator off.\\[1.5ex]
{\em Synopsis}:
\vspace{-0.05in}
\scriptsize
\begin{lstlisting}
   OFF   											
\end{lstlisting}
\normalsize
\vspace{-0.05in}


\section{\quad\code{ON}}\inxx{Commands,{\code{on}}}
\label{manpages:ON}
\label{manpages:on}
\vspace{-0.1in}
{\bf Description}: 	Turn the simulator on.\\[1.5ex]
{\em Synopsis}:
\vspace{-0.05in}
\scriptsize
\begin{lstlisting}
   ON   											
\end{lstlisting}
\normalsize
\vspace{-0.05in}


\section{\quad\code{PARSEOBJDUMP}}\inxx{Commands,{\code{parseobjdump}}}
\label{manpages:PARSEOBJDUMP}
\label{manpages:parseobjdump}
\vspace{-0.1in}
{\bf Description}: 	Parse a GNU objdump file and load into memory.\\[1.5ex]
{\em Synopsis}:
\vspace{-0.05in}
\scriptsize
\begin{lstlisting}
   PARSEOBJDUMP   <objdump file path (string)>																
\end{lstlisting}
\normalsize
\vspace{-0.05in}


\section{\quad\code{PAUINFO}}\inxx{Commands,{\code{pauinfo}}}
\label{manpages:PAUINFO}
\label{manpages:pauinfo}
\vspace{-0.1in}
{\bf Description}: 	Show information about all valid PAU entries.\\[1.5ex]
{\em Synopsis}:
\vspace{-0.05in}
\scriptsize
\begin{lstlisting}
   PAUINFO   								
\end{lstlisting}
\normalsize
\vspace{-0.05in}


\section{\quad\code{PAUSE}}\inxx{Commands,{\code{pause}}}
\label{manpages:PAUSE}
\label{manpages:pause}
\vspace{-0.1in}
{\bf Description}: 	Pause the simulation for arg seconds.\\[1.5ex]
{\em Synopsis}:
\vspace{-0.05in}
\scriptsize
\begin{lstlisting}
   PAUSE   <duration of pause in seconds (integer)>					
\end{lstlisting}
\normalsize
\vspace{-0.05in}


\section{\quad\code{PCBT}}\inxx{Commands,{\code{pcbt}}}
\label{manpages:PCBT}
\label{manpages:pcbt}
\vspace{-0.1in}
{\bf Description}: 	Dump PC backtrace.\\[1.5ex]
{\em Synopsis}:
\vspace{-0.05in}
\scriptsize
\begin{lstlisting}
   PCBT   																							
\end{lstlisting}
\normalsize
\vspace{-0.05in}


\section{\quad\code{PD}}\inxx{Commands,{\code{pd}}}
\label{manpages:PD}
\label{manpages:pd}
\vspace{-0.1in}
{\bf Description}: 	Disable simulation of processor's pipeline.\\[1.5ex]
{\em Synopsis}:
\vspace{-0.05in}
\scriptsize
\begin{lstlisting}
   PD   								
\end{lstlisting}
\normalsize
\vspace{-0.05in}


\section{\quad\code{PE}}\inxx{Commands,{\code{pe}}}
\label{manpages:PE}
\label{manpages:pe}
\vspace{-0.1in}
{\bf Description}: 	Enable simulation of processor's pipeline.\\[1.5ex]
{\em Synopsis}:
\vspace{-0.05in}
\scriptsize
\begin{lstlisting}
   PE   									
\end{lstlisting}
\normalsize
\vspace{-0.05in}


\section{\quad\code{PF}}\inxx{Commands,{\code{pf}}}
\label{manpages:PF}
\label{manpages:pf}
\vspace{-0.1in}
{\bf Description}: 	Flush the pipeline.\\[1.5ex]
{\em Synopsis}:
\vspace{-0.05in}
\scriptsize
\begin{lstlisting}
   PF   											
\end{lstlisting}
\normalsize
\vspace{-0.05in}


\section{\quad\code{PFUN}}\inxx{Commands,{\code{pfun}}}
\label{manpages:PFUN}
\label{manpages:pfun}
\vspace{-0.1in}
{\bf Description}: 	Change probability distrib fxn (default is uniform).\\[1.5ex]
{\em Synopsis}:
\vspace{-0.05in}
\scriptsize
\begin{lstlisting}
   PFUN   							
\end{lstlisting}
\normalsize
\vspace{-0.05in}


\section{\quad\code{PI}}\inxx{Commands,{\code{pi}}}
\label{manpages:PI}
\label{manpages:pi}
\vspace{-0.1in}
{\bf Description}: 	Synonym for PAUINFO.\\[1.5ex]
{\em Synopsis}:
\vspace{-0.05in}
\scriptsize
\begin{lstlisting}
   PI   											
\end{lstlisting}
\normalsize
\vspace{-0.05in}


\section{\quad\code{POWERSTATS}}\inxx{Commands,{\code{powerstats}}}
\label{manpages:POWERSTATS}
\label{manpages:powerstats}
\vspace{-0.1in}
{\bf Description}: 	Show estimated energy and circuit activity.\\[1.5ex]
{\em Synopsis}:
\vspace{-0.05in}
\scriptsize
\begin{lstlisting}
   POWERSTATS   								
\end{lstlisting}
\normalsize
\vspace{-0.05in}


\section{\quad\code{POWERTOTAL}}\inxx{Commands,{\code{powertotal}}}
\label{manpages:POWERTOTAL}
\label{manpages:powertotal}
\vspace{-0.1in}
{\bf Description}: 	Print total power accross all node.\\[1.5ex]
{\em Synopsis}:
\vspace{-0.05in}
\scriptsize
\begin{lstlisting}
   POWERTOTAL   																				
\end{lstlisting}
\normalsize
\vspace{-0.05in}


\section{\quad\code{PS}}\inxx{Commands,{\code{ps}}}
\label{manpages:PS}
\label{manpages:ps}
\vspace{-0.1in}
{\bf Description}: 	Synonym for POWERSTATS.\\[1.5ex]
{\em Synopsis}:
\vspace{-0.05in}
\scriptsize
\begin{lstlisting}
   PS   											
\end{lstlisting}
\normalsize
\vspace{-0.05in}


\section{\quad\code{PWD}}\inxx{Commands,{\code{pwd}}}
\label{manpages:PWD}
\label{manpages:pwd}
\vspace{-0.1in}
{\bf Description}: 	Get current working directory.\\[1.5ex]
{\em Synopsis}:
\vspace{-0.05in}
\scriptsize
\begin{lstlisting}
   PWD   																					
\end{lstlisting}
\normalsize
\vspace{-0.05in}


\section{\quad\code{Q}}\inxx{Commands,{\code{q}}}
\label{manpages:Q}
\label{manpages:q}
\vspace{-0.1in}
{\bf Description}: 	Synonym for QUIT.\\[1.5ex]
{\em Synopsis}:
\vspace{-0.05in}
\scriptsize
\begin{lstlisting}
   Q   												
\end{lstlisting}
\normalsize
\vspace{-0.05in}


\section{\quad\code{QUIT}}\inxx{Commands,{\code{quit}}}
\label{manpages:QUIT}
\label{manpages:quit}
\vspace{-0.1in}
{\bf Description}: 	Exit the simulator.\\[1.5ex]
{\em Synopsis}:
\vspace{-0.05in}
\scriptsize
\begin{lstlisting}
   QUIT   											
\end{lstlisting}
\normalsize
\vspace{-0.05in}


\section{\quad\code{RANDPRINT}}\inxx{Commands,{\code{randprint}}}
\label{manpages:RANDPRINT}
\label{manpages:randprint}
\vspace{-0.1in}
{\bf Description}: 	Print a random value from the selected distribution with given parameters.\\[1.5ex]
{\em Synopsis}:
\vspace{-0.05in}
\scriptsize
\begin{lstlisting}
   RANDPRINT   <distribution name (string)> <min (real)> <max (real)> <p1 (real)> <p2 (real)> <p3 (real)> <p4 (real)>			
\end{lstlisting}
\normalsize
\vspace{-0.05in}


\section{\quad\code{RATIO}}\inxx{Commands,{\code{ratio}}}
\label{manpages:RATIO}
\label{manpages:ratio}
\vspace{-0.1in}
{\bf Description}: 	Print ratio of cycles spent active to those spent sleeping.\\[1.5ex]
{\em Synopsis}:
\vspace{-0.05in}
\scriptsize
\begin{lstlisting}
   RATIO   						
\end{lstlisting}
\normalsize
\vspace{-0.05in}


\section{\quad\code{REGISTERRVAR}}\inxx{Commands,{\code{registerrvar}}}
\label{manpages:REGISTERRVAR}
\label{manpages:registerrvar}
\vspace{-0.1in}
{\bf Description}: 	Register a simulator internal implementation variable or structure for periodic updates either overwriting values or summing determined by the mode parameter.\\[1.5ex]
{\em Synopsis}:
\vspace{-0.05in}
\scriptsize
\begin{lstlisting}
   REGISTERRVAR    <sim var name (string)> <index for array structures (integer)> <value dist name (string)> <value dist p1 (real)> <value dist p2 (real)> <value dist p3 (real)> <value dist p4 (real)> <duration dist name (string)> <duration dist p1 (real)> <duration dist p2 (real)> <duration dist p3 (real)> <duration dist p4 (real)> <mode (integer)>	
\end{lstlisting}
\normalsize
\vspace{-0.05in}


\section{\quad\code{REGISTERSTABS}}\inxx{Commands,{\code{registerstabs}}}
\label{manpages:REGISTERSTABS}
\label{manpages:registerstabs}
\vspace{-0.1in}
{\bf Description}: 	Register variables in a STABS file with value tracing framework.\\[1.5ex]
{\em Synopsis}:
\vspace{-0.05in}
\scriptsize
\begin{lstlisting}
   REGISTERSTABS   <STABS filename (string)>														
\end{lstlisting}
\normalsize
\vspace{-0.05in}


\section{\quad\code{RENUMBERNODES}}\inxx{Commands,{\code{renumbernodes}}}
\label{manpages:RENUMBERNODES}
\label{manpages:renumbernodes}
\vspace{-0.1in}
{\bf Description}: 	Renumber nodes based on base node ID.\\[1.5ex]
{\em Synopsis}:
\vspace{-0.05in}
\scriptsize
\begin{lstlisting}
   RENUMBERNODES   																				
\end{lstlisting}
\normalsize
\vspace{-0.05in}


\section{\quad\code{RESETALLCTRS}}\inxx{Commands,{\code{resetallctrs}}}
\label{manpages:RESETALLCTRS}
\label{manpages:resetallctrs}
\vspace{-0.1in}
{\bf Description}: 	Reset simulation rate measurement trip counters for all nodes.\\[1.5ex]
{\em Synopsis}:
\vspace{-0.05in}
\scriptsize
\begin{lstlisting}
   RESETALLCTRS   																	
\end{lstlisting}
\normalsize
\vspace{-0.05in}


\section{\quad\code{RESETCPU}}\inxx{Commands,{\code{resetcpu}}}
\label{manpages:RESETCPU}
\label{manpages:resetcpu}
\vspace{-0.1in}
{\bf Description}: 	Reset entire simulated CPU state.\\[1.5ex]
{\em Synopsis}:
\vspace{-0.05in}
\scriptsize
\begin{lstlisting}
   RESETCPU   										
\end{lstlisting}
\normalsize
\vspace{-0.05in}


\section{\quad\code{RESETNODECTRS}}\inxx{Commands,{\code{resetnodectrs}}}
\label{manpages:RESETNODECTRS}
\label{manpages:resetnodectrs}
\vspace{-0.1in}
{\bf Description}: 	Reset simulation rate measurement trip counters for current node only.\\[1.5ex]
{\em Synopsis}:
\vspace{-0.05in}
\scriptsize
\begin{lstlisting}
   RESETNODECTRS   																
\end{lstlisting}
\normalsize
\vspace{-0.05in}


\section{\quad\code{RETRYALG}}\inxx{Commands,{\code{retryalg}}}
\label{manpages:RETRYALG}
\label{manpages:retryalg}
\vspace{-0.1in}
{\bf Description}: 	set NIC retransmission backoff algorithm.\\[1.5ex]
{\em Synopsis}:
\vspace{-0.05in}
\scriptsize
\begin{lstlisting}
   RETRYALG   <ifc # (integer)> <algname (string)>																
\end{lstlisting}
\normalsize
\vspace{-0.05in}


\section{\quad\code{R}}\inxx{Commands,{\code{r}}}
\label{manpages:R}
\label{manpages:r}
\vspace{-0.1in}
{\bf Description}: 	Synonym for RATIO.\\[1.5ex]
{\em Synopsis}:
\vspace{-0.05in}
\scriptsize
\begin{lstlisting}
   R   												
\end{lstlisting}
\normalsize
\vspace{-0.05in}


\section{\quad\code{RUN}}\inxx{Commands,{\code{run}}}
\label{manpages:RUN}
\label{manpages:run}
\vspace{-0.1in}
{\bf Description}: 	Mark a node as runnable.\\[1.5ex]
{\em Synopsis}:
\vspace{-0.05in}
\scriptsize
\begin{lstlisting}
   RUN   											
\end{lstlisting}
\normalsize
\vspace{-0.05in}


\section{\quad\code{SAVE}}\inxx{Commands,{\code{save}}}
\label{manpages:SAVE}
\label{manpages:save}
\vspace{-0.1in}
{\bf Description}: 	Dump memory region to disk.\\[1.5ex]
{\em Synopsis}:
\vspace{-0.05in}
\scriptsize
\begin{lstlisting}
   SAVE   <start mem addr (hexadecimal)> <end mem addr (hexadecimal)> <filename (string)>	
\end{lstlisting}
\normalsize
\vspace{-0.05in}


\section{\quad\code{SENSORSDEBUG}}\inxx{Commands,{\code{sensorsdebug}}}
\label{manpages:SENSORSDEBUG}
\label{manpages:sensorsdebug}
\vspace{-0.1in}
{\bf Description}: 	Display various statistics on sensors and signals.\\[1.5ex]
{\em Synopsis}:
\vspace{-0.05in}
\scriptsize
\begin{lstlisting}
   SENSORSDEBUG    																		
\end{lstlisting}
\normalsize
\vspace{-0.05in}


\section{\quad\code{SETBASENODEID}}\inxx{Commands,{\code{setbasenodeid}}}
\label{manpages:SETBASENODEID}
\label{manpages:setbasenodeid}
\vspace{-0.1in}
{\bf Description}: 	Set ID of first node from which all node IDs will be offset.\\[1.5ex]
{\em Synopsis}:
\vspace{-0.05in}
\scriptsize
\begin{lstlisting}
   SETBASENODEID   <base (integer)>																
\end{lstlisting}
\normalsize
\vspace{-0.05in}


\section{\quad\code{SETBATTFEEDPERIOD}}\inxx{Commands,{\code{setbattfeedperiod}}}
\label{manpages:SETBATTFEEDPERIOD}
\label{manpages:setbattfeedperiod}
\vspace{-0.1in}
{\bf Description}: 	Set update periodicity for battery simulation.\\[1.5ex]
{\em Synopsis}:
\vspace{-0.05in}
\scriptsize
\begin{lstlisting}
   SETBATTFEEDPERIOD   <period in picoseconds (integer)>															
\end{lstlisting}
\normalsize
\vspace{-0.05in}


\section{\quad\code{SETBATT}}\inxx{Commands,{\code{setbatt}}}
\label{manpages:SETBATT}
\label{manpages:setbatt}
\vspace{-0.1in}
{\bf Description}: 	Set current battery.\\[1.5ex]
{\em Synopsis}:
\vspace{-0.05in}
\scriptsize
\begin{lstlisting}
   SETBATT   <Battery ID (integer)>																				
\end{lstlisting}
\normalsize
\vspace{-0.05in}


\section{\quad\code{SETDUMPPWRPERIOD}}\inxx{Commands,{\code{setdumppwrperiod}}}
\label{manpages:SETDUMPPWRPERIOD}
\label{manpages:setdumppwrperiod}
\vspace{-0.1in}
{\bf Description}: 	Set periodicity power logging to simlog.\\[1.5ex]
{\em Synopsis}:
\vspace{-0.05in}
\scriptsize
\begin{lstlisting}
   SETDUMPPWRPERIOD   <period in picoseconds (integer)>																
\end{lstlisting}
\normalsize
\vspace{-0.05in}


\section{\quad\code{SETFAULTPERIOD}}\inxx{Commands,{\code{setfaultperiod}}}
\label{manpages:SETFAULTPERIOD}
\label{manpages:setfaultperiod}
\vspace{-0.1in}
{\bf Description}: 	Set period for activating fault scheduling.\\[1.5ex]
{\em Synopsis}:
\vspace{-0.05in}
\scriptsize
\begin{lstlisting}
   SETFAULTPERIOD   <period in picoseconds (integer)>																
\end{lstlisting}
\normalsize
\vspace{-0.05in}


\section{\quad\code{SETFLASHRLATENCY}}\inxx{Commands,{\code{setflashrlatency}}}
\label{manpages:SETFLASHRLATENCY}
\label{manpages:setflashrlatency}
\vspace{-0.1in}
{\bf Description}: 	Set flash read latency.\\[1.5ex]
{\em Synopsis}:
\vspace{-0.05in}
\scriptsize
\begin{lstlisting}
   SETFLASHRLATENCY    <latency in clock cycles (integer)>																		
\end{lstlisting}
\normalsize
\vspace{-0.05in}


\section{\quad\code{SETFLASHWLATENCY}}\inxx{Commands,{\code{setflashwlatency}}}
\label{manpages:SETFLASHWLATENCY}
\label{manpages:setflashwlatency}
\vspace{-0.1in}
{\bf Description}: 	Set flash write latency.\\[1.5ex]
{\em Synopsis}:
\vspace{-0.05in}
\scriptsize
\begin{lstlisting}
   SETFLASHWLATENCY    <latency in clock cycles (integer)>																		
\end{lstlisting}
\normalsize
\vspace{-0.05in}


\section{\quad\code{SETFREQ}}\inxx{Commands,{\code{setfreq}}}
\label{manpages:SETFREQ}
\label{manpages:setfreq}
\vspace{-0.1in}
{\bf Description}: 	Set operating frequency from voltage.\\[1.5ex]
{\em Synopsis}:
\vspace{-0.05in}
\scriptsize
\begin{lstlisting}
   SETFREQ   <freq/MHz (real)>								
\end{lstlisting}
\normalsize
\vspace{-0.05in}


\section{\quad\code{SETIFCOUI}}\inxx{Commands,{\code{setifcoui}}}
\label{manpages:SETIFCOUI}
\label{manpages:setifcoui}
\vspace{-0.1in}
{\bf Description}: 	Set OUI for current IFC.\\[1.5ex]
{\em Synopsis}:
\vspace{-0.05in}
\scriptsize
\begin{lstlisting}
   SETIFCOUI   <which IFC (integer)> <new OUI (integer)>																	
\end{lstlisting}
\normalsize
\vspace{-0.05in}


\section{\quad\code{SETLOC}}\inxx{Commands,{\code{setloc}}}
\label{manpages:SETLOC}
\label{manpages:setloc}
\vspace{-0.1in}
{\bf Description}: 	Set or change node location.\\[1.5ex]
{\em Synopsis}:
\vspace{-0.05in}
\scriptsize
\begin{lstlisting}
   SETLOC   <xloc (real)> <yloc>  (real) <zloc (real)>																	
\end{lstlisting}
\normalsize
\vspace{-0.05in}


\section{\quad\code{SETMEMBASE}}\inxx{Commands,{\code{setmembase}}}
\label{manpages:SETMEMBASE}
\label{manpages:setmembase}
\vspace{-0.1in}
{\bf Description}: 	Set base address of simulator memorry array.\\[1.5ex]
{\em Synopsis}:
\vspace{-0.05in}
\scriptsize
\begin{lstlisting}
   SETMEMBASE   <address (integer)>						
\end{lstlisting}
\normalsize
\vspace{-0.05in}


\section{\quad\code{SETMEMRLATENCY}}\inxx{Commands,{\code{setmemrlatency}}}
\label{manpages:SETMEMRLATENCY}
\label{manpages:setmemrlatency}
\vspace{-0.1in}
{\bf Description}: 	Set memory read latency.\\[1.5ex]
{\em Synopsis}:
\vspace{-0.05in}
\scriptsize
\begin{lstlisting}
   SETMEMRLATENCY    <latency in clock cycles (integer)>																		
\end{lstlisting}
\normalsize
\vspace{-0.05in}


\section{\quad\code{SETMEMWLATENCY}}\inxx{Commands,{\code{setmemwlatency}}}
\label{manpages:SETMEMWLATENCY}
\label{manpages:setmemwlatency}
\vspace{-0.1in}
{\bf Description}: 	Set memory write latency.\\[1.5ex]
{\em Synopsis}:
\vspace{-0.05in}
\scriptsize
\begin{lstlisting}
   SETMEMWLATENCY    <latency in clock cycles (integer)>																		
\end{lstlisting}
\normalsize
\vspace{-0.05in}


\section{\quad\code{SETNETPERIOD}}\inxx{Commands,{\code{setnetperiod}}}
\label{manpages:SETNETPERIOD}
\label{manpages:setnetperiod}
\vspace{-0.1in}
{\bf Description}: 	Set period for activting network scheduling.\\[1.5ex]
{\em Synopsis}:
\vspace{-0.05in}
\scriptsize
\begin{lstlisting}
   SETNETPERIOD   <period in picoseconds (integer)>																
\end{lstlisting}
\normalsize
\vspace{-0.05in}


\section{\quad\code{SETNODEMASS}}\inxx{Commands,{\code{setnodemass}}}
\label{manpages:SETNODEMASS}
\label{manpages:setnodemass}
\vspace{-0.1in}
{\bf Description}: 	Set node mass.\\[1.5ex]
{\em Synopsis}:
\vspace{-0.05in}
\scriptsize
\begin{lstlisting}
   SETNODEMASS    <mass in kg>																						
\end{lstlisting}
\normalsize
\vspace{-0.05in}


\section{\quad\code{SETNODE}}\inxx{Commands,{\code{setnode}}}
\label{manpages:SETNODE}
\label{manpages:setnode}
\vspace{-0.1in}
{\bf Description}: 	Set the current simulated node.\\[1.5ex]
{\em Synopsis}:
\vspace{-0.05in}
\scriptsize
\begin{lstlisting}
   SETNODE   <node id (integer)>								
\end{lstlisting}
\normalsize
\vspace{-0.05in}


\section{\quad\code{SETPC}}\inxx{Commands,{\code{setpc}}}
\label{manpages:SETPC}
\label{manpages:setpc}
\vspace{-0.1in}
{\bf Description}: 	Set the value of the program counter.\\[1.5ex]
{\em Synopsis}:
\vspace{-0.05in}
\scriptsize
\begin{lstlisting}
   SETPC   <PC value (integer)>							
\end{lstlisting}
\normalsize
\vspace{-0.05in}


\section{\quad\code{SETPHYSICSPERIOD}}\inxx{Commands,{\code{setphysicsperiod}}}
\label{manpages:SETPHYSICSPERIOD}
\label{manpages:setphysicsperiod}
\vspace{-0.1in}
{\bf Description}: 	Set update periodicity for physical phenomenon simulation.\\[1.5ex]
{\em Synopsis}:
\vspace{-0.05in}
\scriptsize
\begin{lstlisting}
   SETPHYSICSPERIOD   <period in picoseconds (integer)>														
\end{lstlisting}
\normalsize
\vspace{-0.05in}


\section{\quad\code{SETPROPULSIONCOEFFS}}\inxx{Commands,{\code{setpropulsioncoeffs}}}
\label{manpages:SETPROPULSIONCOEFFS}
\label{manpages:setpropulsioncoeffs}
\vspace{-0.1in}
{\bf Description}: 	Set propulsion power model coefficients.\\[1.5ex]
{\em Synopsis}:
\vspace{-0.05in}
\scriptsize
\begin{lstlisting}
   SETPROPULSIONCOEFFS    <xk1 xk2 xk3 xk4 xk5 xk6 yk1 yk2 yk3 yk4 yk5 yk6 zk1 zk2 zk3 zk4 zk5 zk6>											
\end{lstlisting}
\normalsize
\vspace{-0.05in}


\section{\quad\code{SETQUANTUM}}\inxx{Commands,{\code{setquantum}}}
\label{manpages:SETQUANTUM}
\label{manpages:setquantum}
\vspace{-0.1in}
{\bf Description}: 	Set simulation instruction group quantum.\\[1.5ex]
{\em Synopsis}:
\vspace{-0.05in}
\scriptsize
\begin{lstlisting}
   SETQUANTUM   <quantum (integer)>																		
\end{lstlisting}
\normalsize
\vspace{-0.05in}


\section{\quad\code{SETRANDOMSEED}}\inxx{Commands,{\code{setrandomseed}}}
\label{manpages:SETRANDOMSEED}
\label{manpages:setrandomseed}
\vspace{-0.1in}
{\bf Description}: 	Reinitialize random number generation system with a specific seed useful in conjunction with GETRANDOMSEED for reproducing same pseudorandom state.\\[1.5ex]
{\em Synopsis}:
\vspace{-0.05in}
\scriptsize
\begin{lstlisting}
   SETRANDOMSEED   <seed value negative one to use current time (integer)>	
\end{lstlisting}
\normalsize
\vspace{-0.05in}


\section{\quad\code{SETSCALEALPHA}}\inxx{Commands,{\code{setscalealpha}}}
\label{manpages:SETSCALEALPHA}
\label{manpages:setscalealpha}
\vspace{-0.1in}
{\bf Description}: 	Set technology alpha parameter for use in voltage scaling.\\[1.5ex]
{\em Synopsis}:
\vspace{-0.05in}
\scriptsize
\begin{lstlisting}
   SETSCALEALPHA   <Sakurai alpha (real)>															
\end{lstlisting}
\normalsize
\vspace{-0.05in}


\section{\quad\code{SETSCALEK}}\inxx{Commands,{\code{setscalek}}}
\label{manpages:SETSCALEK}
\label{manpages:setscalek}
\vspace{-0.1in}
{\bf Description}: 	Set technology K parameter for use in voltage scaling.\\[1.5ex]
{\em Synopsis}:
\vspace{-0.05in}
\scriptsize
\begin{lstlisting}
   SETSCALEK   <Sakurai K (real)>																
\end{lstlisting}
\normalsize
\vspace{-0.05in}


\section{\quad\code{SETSCALEVT}}\inxx{Commands,{\code{setscalevt}}}
\label{manpages:SETSCALEVT}
\label{manpages:setscalevt}
\vspace{-0.1in}
{\bf Description}: 	Set technology Vt for use in voltage scaling.\\[1.5ex]
{\em Synopsis}:
\vspace{-0.05in}
\scriptsize
\begin{lstlisting}
   SETSCALEVT   <Vt (real)>																		
\end{lstlisting}
\normalsize
\vspace{-0.05in}


\section{\quad\code{SETSCHEDRANDOM}}\inxx{Commands,{\code{setschedrandom}}}
\label{manpages:SETSCHEDRANDOM}
\label{manpages:setschedrandom}
\vspace{-0.1in}
{\bf Description}: 	Use a different random order for node simulation every cycle.\\[1.5ex]
{\em Synopsis}:
\vspace{-0.05in}
\scriptsize
\begin{lstlisting}
   SETSCHEDRANDOM   																	
\end{lstlisting}
\normalsize
\vspace{-0.05in}


\section{\quad\code{SETSCHEDROUNDROBIN}}\inxx{Commands,{\code{setschedroundrobin}}}
\label{manpages:SETSCHEDROUNDROBIN}
\label{manpages:setschedroundrobin}
\vspace{-0.1in}
{\bf Description}: 	Use a round-robin order for node simulation.\\[1.5ex]
{\em Synopsis}:
\vspace{-0.05in}
\scriptsize
\begin{lstlisting}
   SETSCHEDROUNDROBIN   																			
\end{lstlisting}
\normalsize
\vspace{-0.05in}


\section{\quad\code{SETTIMERDELAY}}\inxx{Commands,{\code{settimerdelay}}}
\label{manpages:SETTIMERDELAY}
\label{manpages:settimerdelay}
\vspace{-0.1in}
{\bf Description}: 	Change granularity of timer intrs.\\[1.5ex]
{\em Synopsis}:
\vspace{-0.05in}
\scriptsize
\begin{lstlisting}
   SETTIMERDELAY   <granularity in microseconds (integer)>																
\end{lstlisting}
\normalsize
\vspace{-0.05in}


\section{\quad\code{SETVDD}}\inxx{Commands,{\code{setvdd}}}
\label{manpages:SETVDD}
\label{manpages:setvdd}
\vspace{-0.1in}
{\bf Description}: 	Set operating voltage from frequency.\\[1.5ex]
{\em Synopsis}:
\vspace{-0.05in}
\scriptsize
\begin{lstlisting}
   SETVDD   <Vdd/volts (real)>							
\end{lstlisting}
\normalsize
\vspace{-0.05in}


\section{\quad\code{SFATAL}}\inxx{Commands,{\code{sfatal}}}
\label{manpages:SFATAL}
\label{manpages:sfatal}
\vspace{-0.1in}
{\bf Description}: 	Induce a node death and state dump.\\[1.5ex]
{\em Synopsis}:
\vspace{-0.05in}
\scriptsize
\begin{lstlisting}
   SFATAL   <suicide note (string)>																		
\end{lstlisting}
\normalsize
\vspace{-0.05in}


\section{\quad\code{SHAREBUS}}\inxx{Commands,{\code{sharebus}}}
\label{manpages:SHAREBUS}
\label{manpages:sharebus}
\vspace{-0.1in}
{\bf Description}: 	Share bus structure with ther named node.\\[1.5ex]
{\em Synopsis}:
\vspace{-0.05in}
\scriptsize
\begin{lstlisting}
   SHAREBUS   <Bus donor nodeid (integer)>																	
\end{lstlisting}
\normalsize
\vspace{-0.05in}


\section{\quad\code{SHOWCLK}}\inxx{Commands,{\code{showclk}}}
\label{manpages:SHOWCLK}
\label{manpages:showclk}
\vspace{-0.1in}
{\bf Description}: 	Show the number of clock cycles simulated since processor reset.\\[1.5ex]
{\em Synopsis}:
\vspace{-0.05in}
\scriptsize
\begin{lstlisting}
   SHOWCLK   						
\end{lstlisting}
\normalsize
\vspace{-0.05in}


\section{\quad\code{SHOWMEMBASE}}\inxx{Commands,{\code{showmembase}}}
\label{manpages:SHOWMEMBASE}
\label{manpages:showmembase}
\vspace{-0.1in}
{\bf Description}: 	Show base address of simulator memorry array.\\[1.5ex]
{\em Synopsis}:
\vspace{-0.05in}
\scriptsize
\begin{lstlisting}
   SHOWMEMBASE   								
\end{lstlisting}
\normalsize
\vspace{-0.05in}


\section{\quad\code{SHOWPIPE}}\inxx{Commands,{\code{showpipe}}}
\label{manpages:SHOWPIPE}
\label{manpages:showpipe}
\vspace{-0.1in}
{\bf Description}: 	Show contents of the processor pipeline.\\[1.5ex]
{\em Synopsis}:
\vspace{-0.05in}
\scriptsize
\begin{lstlisting}
   SHOWPIPE   									
\end{lstlisting}
\normalsize
\vspace{-0.05in}


\section{\quad\code{SIGSRC}}\inxx{Commands,{\code{sigsrc}}}
\label{manpages:SIGSRC}
\label{manpages:sigsrc}
\vspace{-0.1in}
{\bf Description}: 	Create a physical phenomenon signal source.\\[1.5ex]
{\em Synopsis}:
\vspace{-0.05in}
\scriptsize
\begin{lstlisting}
   SIGSRC   <type (integer)> <description (string)> <tau (real)> <propagationspeed (real)> <A (real)> <B (real)> <C (real)> <D (real)> <E (real)> <F (real)> <G (real)> <H (real)> <I (real)> <J (real)> <K (real)> <m (real)> <n (real)> <o (real)> <p (real)> <q (real)> <r (real)> <s (real)> <t (real)> <x (real)> <y (real)> <z (real)> <trajectoryfile (string)> <trajectoryrate (real)> <looptrajectory (Boolean)> <samplesfile (string)> <samplerate (integer)> <fixedsampleval (real)> <loopsamples (Boolean)>	
\end{lstlisting}
\normalsize
\vspace{-0.05in}


\section{\quad\code{SIGSUBSCRIBE}}\inxx{Commands,{\code{sigsubscribe}}}
\label{manpages:SIGSUBSCRIBE}
\label{manpages:sigsubscribe}
\vspace{-0.1in}
{\bf Description}: 	Subscribe sensor X on the current node to a signal source Y.\\[1.5ex]
{\em Synopsis}:
\vspace{-0.05in}
\scriptsize
\begin{lstlisting}
   SIGSUBSCRIBE   <X (integer)> <Y (integer)>														
\end{lstlisting}
\normalsize
\vspace{-0.05in}


\section{\quad\code{SIZEMEM}}\inxx{Commands,{\code{sizemem}}}
\label{manpages:SIZEMEM}
\label{manpages:sizemem}
\vspace{-0.1in}
{\bf Description}: 	Set the size of memory.\\[1.5ex]
{\em Synopsis}:
\vspace{-0.05in}
\scriptsize
\begin{lstlisting}
   SIZEMEM   <size of memory in bytes (integer)>							
\end{lstlisting}
\normalsize
\vspace{-0.05in}


\section{\quad\code{SIZEPAU}}\inxx{Commands,{\code{sizepau}}}
\label{manpages:SIZEPAU}
\label{manpages:sizepau}
\vspace{-0.1in}
{\bf Description}: 	Set the size of the PAU.\\[1.5ex]
{\em Synopsis}:
\vspace{-0.05in}
\scriptsize
\begin{lstlisting}
   SIZEPAU   <size of PAU in number of entries (integer)>						
\end{lstlisting}
\normalsize
\vspace{-0.05in}


\section{\quad\code{SPLIT}}\inxx{Commands,{\code{split}}}
\label{manpages:SPLIT}
\label{manpages:split}
\vspace{-0.1in}
{\bf Description}: 	Split current CPU to execute from a new PC and stack.\\[1.5ex]
{\em Synopsis}:
\vspace{-0.05in}
\scriptsize
\begin{lstlisting}
   SPLIT   <newpc (hexadecimal)> <newstackaddr (hexadecimal)> <argaddr (hexadecimal)> <newcpuidstr (integer)>						
\end{lstlisting}
\normalsize
\vspace{-0.05in}


\section{\quad\code{SRECL}}\inxx{Commands,{\code{srecl}}}
\label{manpages:SRECL}
\label{manpages:srecl}
\vspace{-0.1in}
{\bf Description}: 	Load a binary program in Motorola S-Record format.\\[1.5ex]
{\em Synopsis}:
\vspace{-0.05in}
\scriptsize
\begin{lstlisting}
   SRECL   								
\end{lstlisting}
\normalsize
\vspace{-0.05in}


\section{\quad\code{STOP}}\inxx{Commands,{\code{stop}}}
\label{manpages:STOP}
\label{manpages:stop}
\vspace{-0.1in}
{\bf Description}: 	Mark the current node as unrunnable.\\[1.5ex]
{\em Synopsis}:
\vspace{-0.05in}
\scriptsize
\begin{lstlisting}
   STOP   																				
\end{lstlisting}
\normalsize
\vspace{-0.05in}


\section{\quad\code{THROTTLE}}\inxx{Commands,{\code{throttle}}}
\label{manpages:THROTTLE}
\label{manpages:throttle}
\vspace{-0.1in}
{\bf Description}: 	Set the throttling delay in nanoseconds.\\[1.5ex]
{\em Synopsis}:
\vspace{-0.05in}
\scriptsize
\begin{lstlisting}
   THROTTLE   <throttle delay in nanoseconds (integer)>															
\end{lstlisting}
\normalsize
\vspace{-0.05in}


\section{\quad\code{THROTTLEWIN}}\inxx{Commands,{\code{throttlewin}}}
\label{manpages:THROTTLEWIN}
\label{manpages:throttlewin}
\vspace{-0.1in}
{\bf Description}: 	Set the throttling window --- main simulation loop sleeps for throttlensecs x throttlewin nanosecs every throttlewin simulation cycles\\[1.5ex]
{\em Synopsis}:
\vspace{-0.05in}
\scriptsize
\begin{lstlisting}
   THROTTLEWIN    for an average of throttlensecs sleep per simulation cycle.
\end{lstlisting}
\normalsize
\vspace{-0.05in}


\section{\quad\code{TRACE}}\inxx{Commands,{\code{trace}}}
\label{manpages:TRACE}
\label{manpages:trace}
\vspace{-0.1in}
{\bf Description}: 	Toggle Tracing.\\[1.5ex]
{\em Synopsis}:
\vspace{-0.05in}
\scriptsize
\begin{lstlisting}
   TRACE   																							
\end{lstlisting}
\normalsize
\vspace{-0.05in}


\section{\quad\code{VALUESTATS}}\inxx{Commands,{\code{valuestats}}}
\label{manpages:VALUESTATS}
\label{manpages:valuestats}
\vspace{-0.1in}
{\bf Description}: 	Print data value tracking statistics.\\[1.5ex]
{\em Synopsis}:
\vspace{-0.05in}
\scriptsize
\begin{lstlisting}
   VALUESTATS   																				
\end{lstlisting}
\normalsize
\vspace{-0.05in}


\section{\quad\code{VERBOSE}}\inxx{Commands,{\code{verbose}}}
\label{manpages:VERBOSE}
\label{manpages:verbose}
\vspace{-0.1in}
{\bf Description}: 	Enable the various prints.\\[1.5ex]
{\em Synopsis}:
\vspace{-0.05in}
\scriptsize
\begin{lstlisting}
   VERBOSE   																						
\end{lstlisting}
\normalsize
\vspace{-0.05in}


\section{\quad\code{VERSION}}\inxx{Commands,{\code{version}}}
\label{manpages:VERSION}
\label{manpages:version}
\vspace{-0.1in}
{\bf Description}: 	Display the simulator version and build.\\[1.5ex]
{\em Synopsis}:
\vspace{-0.05in}
\scriptsize
\begin{lstlisting}
   VERSION   																				
\end{lstlisting}
\normalsize
\vspace{-0.05in}


\section{\quad\code{V}}\inxx{Commands,{\code{v}}}
\label{manpages:V}
\label{manpages:v}
\vspace{-0.1in}
{\bf Description}: 	Synonym for VERBOSE.\\[1.5ex]
{\em Synopsis}:
\vspace{-0.05in}
\scriptsize
\begin{lstlisting}
   V   																						
\end{lstlisting}
\normalsize
\vspace{-0.05in}

