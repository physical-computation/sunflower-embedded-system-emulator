\section{Running \RAY}
\index{running}
  Since \RAY\ runs on a wide variety of platforms, the exact commands
required to run it vary substantially.  The easiest way to get started
using \RAY\ is to try running one of the non-parallel, uniprocessor
versions first.  \RAY\ includes a built-in help page describing all
available command line options with very brief text, this help page is
displayed when \RAY\ is run with the {\tt -help} option.

\index{command line parameters}

\subsection{General command line options}
Several command line options are available to tune 
\RAY\ performance, display built-in help text, and 
set output verbosity.
\begin{itemize}
\item{{\tt -nobounding}}: disable automatic generation of hierarchical
      grid-based acceleration data structures
\item{{\tt -boundthresh {\it object\_count}}}: override default threshold for 
      subdividing grid cells with a new grid
\item{{\tt -numthreads {\it thread\_count}}}: command line override for the number of 
      threads to spawn during the ray tracing process.  When this
      options is not specified, \RAY\ determines the number of threads
      to spawn based on the number of CPUs available on a given node.
\item{{\tt +V}}: enable verbose status messages, including reporting of
      overall node and processor count.
\item{{\tt -nosave}}: disable saving of output images to disk files.  This 
      feature is normally only used when benchmarking, or when using one of
      the OpenGL-enabled \RAY\ configurations which provide runtime display
      of rendered images.
\item{{\tt -camfile {\it filename}}}: run a fly-through animation from 
      the named camera file. 
\end{itemize}

\subsection{Command line shading controls}
\RAY\ supports a number of command line parameters which affect the
quality and algorithms used to render scene files.  The parameters 
select one of several quality levels, which implement various 
compromises between rendering speed and quality.  Along with the
overall shading quality controls, several specific options provide 
control over individual rendering algorithms within \RAY.

\begin{itemize}
\item{{\tt -lowestshade}}: solid colors only
\item{{\tt -lowshade}}: minmalistic shading, using texture colors only
\item{{\tt -mediumshade}}: disables computation of shadows
\item{{\tt -fullshade}}: enables the highest quality rendering mode
\item{{\tt -aasamples {\it sample_count}}: command line override for the number of 
      antialiasing supersamples computed for each pixel.  A value of
      zero disables antialiasing.  If this option is not used, the number
      of antialiasing samples is determined by the contents of the scen file.
\item{{\tt -shade\_phong}}: use traditional phong shading for specular
      highlights.
\item{{\tt -shade\_blinn}}: use Blinn's equation for specular highlights.
\item{{\tt -shade\_blinn\_fast}}: use a fast approximation to Blinn-style
      specular highlights.
\item{{\tt -shade\_nullphong}}: entirely disables computation of
      specular highlights by registering a no-op function pointer.
\item{{\tt -trans\_orig}}: use original \RAY\ transparency mode.
\item{{\tt -trans\_vmd}}: a special transparency mode designed for
      use with VMD.  The resulting color is multiplied by opacity, 
      giving results similar to what one would see with screen-door
      transparency in OpenGL.
\end{itemize}


\subsection{Command line image format options}
\RAY\ optionally supports several image file formats for output.
The output format is specified by the {\tt -format {\it formatname}} 
command line parameter.  Several of these formats are only available if
\RAY\ has been compiled with optional features enabled.

\begin{itemize}
\item{{\tt TARGA}}: uncompressed 24-bit Targa file
\item{{\tt BMP}}: uncompressed 24-bit Windows bitmap 
\item{{\tt PPM}}: uncompressed 24-bit NetPBM portable pixmap (PPM) file
\item{{\tt RGB}}: uncompressed 24-bit Silicon Graphics RGB file
\item{{\tt JPEG}}: compressed 24-bit JPEG file
\item{{\tt PNG}}: uncompressed 24-bit PNG file
\end{itemize}


\subsection{Tips for running MPI versions}
\index{running with MPI}
\RAY\ support the use of MPI for distributed memory rendering
of complex scenes.  Most commercial supercomputers and cluster
vendors provide their own custom-tuned implementations of MPI
which perform optimally on their hardware.  Homegrown clusters
typically use either the LAM or MPICH implementation of MPI.
While \RAY\ will work with any comformant implementation of MPI,
some implementations perform much better than others.  In the
author's experience, the LAM implementation of MPI gives the
best performance when used with \RAY.

\subsection{Interactive ray tracing}
\index{interactive ray tracing}
\RAY\ is fast enough to support ray tracing at interactive rates
when run on a large enough parallel computer, or with a simple enough
scene.  To this end, \RAY\ can be optionally compiled with support for
the Spaceball 6DOF motion control device.  Using the Spaceball, one
can fly around in an otherwise static scene.  This is accomplished
with the {\tt -spaceball {\it serial_port_device}} command line parameters.
