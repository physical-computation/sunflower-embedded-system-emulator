\section{Introduction}
\RAY\ is designed to be a very fast renderer, based on ray tracing,
and employing parallel processing to achieve high performance.

At the present time, \RAY\ and its scene description language are fairly
primitive, this will be remedied as time passes.  For now I'm going to 
skip the ``intro to ray tracing'' and related things that should probably 
go here, they are better addressed by the numerous books on the subject
written by others.  This document is designed to serve the needs of 
sophisticated users that are already experienced with ray tracing
and basic graphics concepts, rather than catering to beginners.  
If you have suggestions for improving this manual, I'll be glad 
to address them as time permits.

Until this document is finished and all-inclusive, the best way to 
learn how \RAY\ works is to examine some of the sample scenes that 
I've included in the \RAY\ distribution.  Although they are all very 
simple, each of the scenes tries to show something slightly 
different \RAY\ can do.  Since \RAY\ is rapidly changing to accommodate 
new rendering primitives and speed optimizations, the scene 
description language is likely to change to some degree as well. 

\subsection{\RAY\ Feature List}
  Although \RAY\ is a relatively simple renderer, it does have enough 
features that they bear some discussion.

\begin{itemize}
\item Parallel execution using MPI.
\item Parallel execution using POSIX or Unix-International 
      threads libraries.
\item Automatic grid-based spatial decomposition scheme for greatly
      increased rendering speeds.
\item Simple antialiasing based on psuedo-random supersampling.
\item Linear, exponential, and exponential-squared fog.
\item Perspective, orthographic, and depth-of-field camera projection
      modes, with eye-space frustum controls.
\item Positional, directional, and spot lights, with optional attenuation.
\item Provides many useful geometric objects including 
      Spheres, Planes, Triangles, Cylinders, Quadrics, and Rings
\item Texture mapping, with automatic MIP-map generation
\item Supports rendering of volumetric data sets 
\end{itemize}


